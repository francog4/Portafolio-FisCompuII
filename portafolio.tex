% Formato de portafolio
% Documento adaptado de https://github.com/PlasmaPhysicsUdeC/FondecyTeX por Roberto Navarro <roberto.navarro@udec.cl> 

% Este documento es modular, es decir, esta separado en varios
% archivos para facilitar su uso. Ver uso del paquete `subfiles`.
% https://www.overleaf.com/learn/latex/Multi-file_LaTeX_projects

\documentclass[twoside]{report}

\usepackage[paper=letterpaper,left=3cm,right=3cm,top=3cm,bottom=2cm,includefoot]{geometry}
\usepackage{fancyhdr}
\usepackage{tabularx}
\usepackage{multirow}
\usepackage[colorlinks]{hyperref}
\usepackage{hhline}
\usepackage{amsmath, amsfonts}
\usepackage{enumitem}
\setenumerate{itemsep=-3pt,topsep=3pt}
\setdescription{itemsep=-3pt,topsep=3pt,leftmargin=!,labelwidth=4.5cm}
\usepackage{pgfgantt}
\usepackage{graphicx}   \graphicspath{{./img/} {./tex/img/}}
\usepackage{float}
\usepackage{caption}
\usepackage{subcaption}

\usepackage[spanish]{babel}
\usepackage[utf8]{inputenc}
\usepackage[T1]{fontenc}

%\usepackage[table]{xcolor}
\usepackage{xcolor}
\definecolor{LightGray}{gray}{0.92}

\usepackage{minted}
\newminted{python}{ linenos,breaklines,mathescape,texcomments,xleftmargin=\parindent, numbersep=8pt,  bgcolor=LightGray}

\renewcommand\listingscaption{C\'odigo}



% bibliografia: descomente estas dos lineas para usar estilo numerico, e.g. [1].
\usepackage[square,numbers,sort&compress]{natbib}
%\bibliographystyle{apsrev4-1}

% bibliografia: descomente estas dos lineas para usar estilo autor-año, e.g. (Navarro, 2022).
% \usepackage[authoryear]{natbib}
% \bibliographystyle{aipauth4-1}


% subfiles permite que el documento sea modular
\usepackage{subfiles} 

 %Comente/Descomente las siguientes lineas para cambiar la fuente del texto
%\usepackage{DejaVuSans}
 %\renewcommand*\familydefault{\sfdefault}
 %\usepackage{sansmath}
% \sansmath



\pagestyle{fancy}
\renewcommand{\footrulewidth}{0.4pt}
\renewcommand{\headrulewidth}{0.4pt}
\fancyfoot{}
\fancyfoot[RE,RO]{\thepage}
\fancyhead[LO,LE]{\textcolor[RGB]{127,127,127}{Portafolio - Física Computacional II (2022)}} 

\definecolor{tcc}{RGB}{217,217,217} % Table cell color

\renewcommand\tabularxcolumn[1]{m{#1}}
\setlength{\arrayrulewidth}{0.5pt}
\renewcommand{\arraystretch}{2}

% \renewcommand{\thesection}{\alph{section})}
% \renewcommand{\thesubsection}{\alph{section}.\arabic{subsection}}

\begin{document}

\subfile{tex/portada}


\tableofcontents



%%% con \subfile se incluyen archivos externos
\subfile{tex/introduccion}
\subfile{tex/presentacion-estudiante}

\chapter{Actividades de laboratorio}  % cuerpo del portafolio
\subfile{tex/0819-derivada-numerica}

%\clearpage
%\subfile{tex/0826-git}

\clearpage
\subfile{tex/0902-integrales}

\clearpage
\subfile{tex/0909-ceros-de-funciones}

\clearpage
\subfile{tex/0923-ec-diferenciales-ordinarias}

\clearpage
\subfile{tex/0930-Runge-Kutta} 

\clearpage
\subfile{tex/1028-Interpolacion}

\clearpage
\subfile{tex/1118-numeros-aleatorios}

\clearpage
\subfile{tex/conclusion}


% lista de referencias guardadas en referencias.bib
\bibliographystyle{amsplain}
\bibliography{referencias}


%\bibitem[1]{equilibrio} @misc{wikistatic,
%    author = "{Wikipedia contributors}",
%    title = "Equilibrio mecánico --- {Wikipedia}{,} The Free %Encyclopedia",
%    year = "2021",
%    howpublished = {\url{https://es.wikipedia.org/wiki/Equilibrio_mecánico#Estabilidad_del_equilibrio}},
%    note = "[Online; accessed 5-November-2021]"
%  }


\end{document}
