\documentclass[../portafolio.tex]{subfiles}

% Solo agregue paquetes en el preámbulo de ../portafolio.tex

\begin{document}

\chapter{Presentación del estudiante}


%%%%%%%%%%%%%%%%%%%%%%%%%%%%%%%%%%%%%%%%%%%%%%%%%%%%%%%%%%%%%%%%%%%%%%%%%%%%%%%%%
\section*{Datos personales}
\begin{description}
\item[{Nombre completo}] Franco Nicolás Giovine López  %coloque sus nombres y apellidos completos, respetando tildes, mayúsculas y minúsculas
\item[{Matrícula}] 2021431560    % matrícula udec
\item[{Fecha de Nacimiento}] 07 de Junio de 2002   % día de mes de año
\item[{Nacionalidad}] Chilena
\end{description}


%%%%%%%%%%%%%%%%%%%%%%%%%%%%%%%%%%%%%%%%%%%%%%%%%%%%%%%%%%%%%%%%%%%%%%%%%%%%%%%%%
\section*{Breve biografía académica}
Soy Franco, estudiante de segundo año de la carrera de Cs. Físicas. La educación media la realicé en el colegio de los Sagrados corazones de
 la ciudad de Concepción. He vivido toda mi vida en Hualpén y mi comida favorita es el pastel de choclo.


%%%%%%%%%%%%%%%%%%%%%%%%%%%%%%%%%%%%%%%%%%%%%%%%%%%%%%%%%%%%%%%%%%%%%%%%%%%%%%%%%
\section*{Visión general e interés sobre la asignatura}
El ramo de Física Computacional II nos dará un un conocimiento mas sólido sobre la herramientas computacionales que necesitaremos en un futuro cuando nos 
desarrollemos como profesionales. Mi interés en este ramo es reforzar los vagos conocimientos que ya tengo en computación.


%%%%%%%%%%%%%%%%%%%%%%%%%%%%%%%%%%%%%%%%%%%%%%%%%%%%%%%%%%%%%%%%%%%%%%%%%%%%%%%%%
\section*{Resultados esperados de este portafolio}
Espero que a fin de semestre cuando este portafolio esté completo y lo lea  me de cuenta de todo lo que he aprendido durante este curso. También espero poder sentirme orgulloso de los avances que he logrado programando.

\end{document}