\documentclass[../portafolio.tex]{subfiles}

\begin{document}

% No edite este archivo

\chapter*{Introducción}
\addcontentsline{toc}{chapter}{Introducción}
\markboth{Introducción}{Introducción}

Este portafolio deberá incluir muestras o evidencias de las actividades desarrolladas en la asignatura de Física Computacional II (510240), dictada en el segundo semestre de 2022 en el departamento de física, facultad de Ciencias Físicas y Matemáticas, de la Universidad de Concepción.

\medskip

Esta asignatura se enfoca en la resolución de problemas en Física
mediante lenguajes de programación especializados (en este caso
\texttt{python}), contribuyendo a conocimientos generales de
construcción y aplicación de algoritmos computacionales, optimización
de simulaciones numéricas y organización de códigos mediante
herramientas computacionales. Además, la metodología y objetivos del
curso se enmarcan en el proyecto de docencia ``Mejorando el
aprendizaje de competencias en computación científica y la elaboración
de informes que comuniquen resultados científicos para estudiantes de
pregrado de Licenciatura en Física'', financiado por el fondo
concursable Innovando en Red UCO 20102 de la Universidad de
Concepción, cuyo objetivo es mejorar el aprendizaje en competencias de
programación científica utilizando herramientas actuales para el
control de cambios en códigos computacionales, como el uso de Git,
Github y \LaTeX\, y mejorar las competencias en elaboración de
informes académicos, lo cual se espera conseguir con la implementación
de este portafolio.

\medskip

Esta asignatura contribuye al logro de las siguientes competencias del
perfil de egreso ``Aplicar lenguajes de programación avanzados tales
como Fortran, C++, Python, etc., a problemas de la Física'', y se espera obtener los siguientes resultados de aprendizaje:
\begin{enumerate}
\item Aplicar las herramientas computacionales en la resolución numérica de problemas en Física.
\item Generar programas computacionales apoyándose en algoritmos y conceptos de la física matemática y
estadística.
\item Diferenciar, integrar y resolver ecuaciones diferenciales ordinarias, en forma numérica.
\end{enumerate}

\medskip

La evaluación se realizará a través de este portafolio. El profesor
evaluará cada una de las muestras y el conjunto del portafolio a
través de una pauta que se entregará oportunamente. El/la estudiante
deberá auto-evaluar su portafolio y seleccionará un conjunto de
evidencias de este portafolio, las cuales serán expuestas ante sus
compañeros a final del semestre, quienes también lo evaluarán a través
de una rúbrica. El documento de portafolio contendrá una sección de
conclusiones donde autoevaluará su desempeñoa través de una reflexión
final y de sus respuestas a una serie de preguntas que serán
consensuadas con el profesor.

\medskip

Como criterios de evaluación, se considerará:
\begin{enumerate}
\item Adecuación de las muestras incluidas a los resultados del
  aprendizaje y a los contenidos de la asignatura.
  
\item Coherencia de las evidencias con las actividades realizadas en clases.
  
\item Competencias comunicativas: escritura (ortografía, gramática,
  sintaxis, estilo) y presentaciones orales (habla correctamente y con
  claridad; utiliza términos técnicos apropiadamente, etcétera).
\item Presentación: Claridad, limpieza y orden del documento.  
\item Autoevaluación: Seriedad y reflexión de las justificaciones dadas al principio y al final del documento.

\end{enumerate}



\end{document}