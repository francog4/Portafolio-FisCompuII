\documentclass[../portafolio.tex]{subfiles}

\begin{document}

\chapter{Conclusiones}

\hfill \textbf{Fecha de presentación:} 14 de diciembre de 2022

\medskip


%--------------------------------------------------------------------------------
% Resumen de los objetivos de este portafolio

En el presente portafolio se trabajaron en distintos problemas para los cuales se necesitaron conocimientos en cálculo numérico y en programación para desarrollar programas computacionales capaces de resolver derivadas, integrales, ecuaciones diferenciales ordinarias, etc. También se trabajaron con herramientas útiles en el área de la computación científica tales como Git, Github y \LaTeX. 

%--------------------------------------------------------------------------------
% Resumen de los contenidos 
% Incluya una breve reflexión de:
% - Lo que aprendió en cada actividad, lo que faltó aprender, lo que no se entendió y lo que sí se entendió bien.
% - Sugerencias para futuras versiones del curso.
% - ¿Cuál es la evidencia de este portafolio que usted cree es mejor/más relevante/en la que aprendió mejor? ¿Qué diferencia a esa evidencia del resto incluido en este portafolio?
% - ¿Se ajustan estas evidencias a sus expectativas originales y a los objetivos del curso?
% - ¿Puede evaluar la utilidad de este portafolio?



\vspace{2mm}
En las actividades de este laboratorio aprendí diversos métodos numéricos para resolver distintos problemas, como por ejemplo: calcular derivadas numéricas, la regla del punto medio para resolver integrales, el método de la bisección para encontrar ceros en funciones trabajando con la energía potencial de un péndulo, el método de Euler-Cromer y el método de Runge-Kutta para resolver ecuaciones diferenciales ordinarias como por ejemplo las ecuaciones de Lotka-Volterra y las ecuaciones de Lorenz, construir polinomios de Lagrange, usar la interpolación spline cúbica y usar la inversa de la matriz de Vandermonde para interpolar datos del promedio de manchas solares por año o usar funciones que generan números aleatorios para encontrar el numero $\pi$ numéricamente. Por otra parte, creo que en el laboratorio de derivadas me faltó aprender cómo calcular los errores, pero lo demás creo haberlo entendido bien. En general me costó entender la demostración de los métodos en la mayoría de las actividades, a excepción de las actividades de integrales, números aleatorios y de ceros de funciones donde creo haber entendido todo.  

\vspace{2mm}
En futuras versiones de este curso recomendaría seguir con la metodología que se estaba siguiendo en las clases de los días viernes, pues siento que las clases se impartían y se desarrollaban de buena manera y que los laboratorios eran útiles para practicar lo aprendido. No obstante, comenzar a trabajar con lo recién enseñado en la segunda clase de los viernes, al menos a mí y a un compañero con el que hablé, se nos hacía un tanto complicado, por ende, creo que dedicar los días miércoles para introducir y/o complementar la primera clase del viernes podría ser más provechoso, ya que personalmente no creo haberle sacado mucho provecho a la clase de los días miércoles. 

\vspace{2mm}
Como ya dije, personalmente creo que en los laboratorios en los que aprendí mejor son los laboratorios de ceros de funciones, integrales y números aleatorios, esto queda evidenciado en el desarrollo de dichos laboratorio, ya que en estos expliqué más detenidamente los pasos de como fueron programados los respectivos métodos, en contraste, en otros laboratorios, como por ejemplo el de Runge-Kutta, solo se señaló en el código cuando se usó el respectivo método y no se explicó a fondo.

\vspace{2mm}
Luego, al comparar las evidencias expuestas en este portafolio con los objetivos iniciales del curso y con las primera ideas que tenía del curso antes de iniciarlo, por una parte creo que los objetivos del curso, tales como aprender a utilizar herramientas computacionales para resolver problemas aplicados a la física, se vieron concretados en el desarrollo de este portafolio y de sus actividades. Por otra parte, al comparar los resultados con mis expectativas iniciales, siento que mis expectativas se vieron sobrepasadas por los resultados, pues pensé que iba a ser un curso menos exigente y en donde iba a aprender menos cosas. En resumen, diría que la creación de este portafolio fue sumamente útil, ya que no solo sirvió para desarrollar las actividades y que estas fueran evaluadas, si no que también sera de gran utilidad en un futuro cuando necesite programar.







%--------------------------------------------------------------------------------
% Autoevaluación del alumno/a
% Realice una reflexión de cómo trabajó usted, qué cree haber hecho bien y mal en el curso, y una reflexión de qué le gustaría  hacer a futuro (en la forma de estudiar y en cómo cree que aplicará los contenidos de este portafolio en el futuro)

\vspace{2mm}
Para finalizar, creo que el trabajo que realicé durante este curso se adecuó a la exigencia que este requería y si pudiera haber cambiado algo en mi método de estudio, creo que hubiese sido estudiar por mi mismo las demostraciones de los métodos enseñados en las clases. Por otra parte, espero recurrir a este portafolio en un futuro de la carrera o ya egresado cuando necesite resolver ciertos problemas que requieran programación.






\end{document}