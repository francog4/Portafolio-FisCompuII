\documentclass[../portafolio.tex]{subfiles}

% Solo agregue paquetes en el preámbulo de ../portafolio.tex

\begin{document}

% En esta sección, explique en detalle los siguientes aspectos:
% - Fecha de realización de la actividad
% - Título de la actividad (dentro de \section)
% - Un párrafo explicando cuál es el objetivo de la actividad
% - Nombre de personas con quien trabajó en la actividad
% - Una selección de evidencias de que usted hizo esta actividad (imágenes, códigos, respuestas a un problema teórico, etc.)
% - Una conclusión breve (qué aprendió con la actividad, qué no entendió, qué faltó trabajar, qué recomienda para futuras sesiones)

% Numero máximo de palabras en esta sección: 1000 palabras.

%%%%%%%%%%%%%%%%%%%%%%%%%%%%%%%%%%%%%%%%%%%%%%%%%%%%%%%%%%%%%%%%%%%%%%%%%%%%%%%%
\section{Introducción a git}   % ejemplo: Derivadas numéricas , introducción a git , 

\hfill \textbf{Fecha de la actividad:} 26 de agosto de 2022

\medskip

%---------------------------------------------------------------------------------
\subsection{Objetivo de la clase}
La finalidad de esta clase fue aprender la utilidad que tiene git y como utilizarlo.

\subsection{Desarrollo del laboratorio}
Lo primero que se hizo fue crear una cuenta de github y luego se abrió la consola para configurar git y comunicar el computador con git. Terminada toda la configuración se descargó el repositorio, se hizo cambios en la presentación del estudiante y finalmente se subieron los cambios.

\subsection{Conclusión}
Personalmente encontré que git, a pesar de ser algo complejo, es una herramienta sumamente útil cuando se necesita trabajar en grupo e incluso cuando se trabaja solo. La secuencia de comandos \texttt{git add}, \texttt{git commit -m "mensaje"} y \texttt{git push} me parecieron bastante simples para los utiles que son.



\end{document}
